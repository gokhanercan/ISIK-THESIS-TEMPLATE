%% ----------------------------------------------------------------
%% Thesis.tex -- MAIN FILE (the one that you compile with LaTeX)
%% ----------------------------------------------------------------

% Set up the document
\documentclass[a4paper, 12pt, oneside]{Thesis}  % Use the "Thesis" style, based on the ECS Thesis style by Steve Gunn

% This is the preamble where you add packages and custom settings.
\usepackage{mathptmx}  % Times New Roman font for pdflatex - Bu Times New Roman'ı scaleable yapıyorumuş.


%\usepackage{fontspec}
%\setmainfont{Times New Roman}

\usepackage{lmodern}  % Load scalable fonts
\usepackage[T1]{fontenc}  % Improve font encoding
\usepackage[english]{babel}         % Language setup
%\usepackage[expansion=false]{microtype}  % Disable font expansion
\usepackage{microtype}              % Improves text justification

%Spacing
\usepackage{setspace}  % For line spacing
\onehalfspacing        % Set 1.5 line spacing

%\usepackage{comment} % To disable parts of the document
%\specialcomment{excludechapter}{}{}

\graphicspath{{figures/}}  % Location of the graphics files (set up for graphics to be in PDF format)

% Bu paketler üniversitenin eski style'ından geliyorlar. Ne yaptıklarını bilmiyorum.
\usepackage{verbatim}
% \usepackage{vector}  
\usepackage{notoccite}
\usepackage{ifthen}

\usepackage[authoryear,round,semicolon]{natbib}		%sort ve square istemiyorum.
%\usepackage[numbers]{natbib}
%\usepackage{cite}

% Türkçe karakter için: https://www.overleaf.com/9545523bkkgxbpwxpgq#/34630773/
\usepackage[utf8]{inputenc} % Required for inputting international characters
\usepackage[T1]{fontenc} % Output font encoding for international characters

\usepackage{tabularx}

%For TODO Management (ex: https://tex.stackexchange.com/questions/9796/how-to-add-todo-notes)
\usepackage{xargs}% Use more than one optional parameter in a new commands

% --------------- TABLES ---------------------------
\usepackage{array}
\newcolumntype{K}[1]{>{\centering\let\newline\\\arraybackslash\hspace{0pt}}m{#1}}
% https://tex.stackexchange.com/questions/127050/vertically-align-text-in-table-in-latex
\usepackage{float}		%https://tex.stackexchange.com/questions/20376/how-to-bind-figure-table-to-a-section

% subfig paketini bir türlü çalıştıramadım, vazgeçtim.
% \usepackage{graphicx}  % include images
% \usepackage[caption=false]{subfig} % subfigures.  false option prevents conflicts in caption styling with other packages

\usepackage{textcomp} %sağa ok yapmak için


\usepackage[margin=1in]{geometry} 
\usepackage{amsmath,amsthm,amssymb}
 
% \newcommand{\N}{\mathbb{N}}
% \newcommand{\Z}{\mathbb{Z}}

\usepackage{harpoon}	%https://tex.stackexchange.com/questions/344778/latex-vector-notation-arrow


\newcommand{\TEZBASLIK}{THESIS TITLE}
% \newcommand{\TEZBASLIK}{SEMANTIC RELATION EXTRACTION FOR TURKISH ENRICHING WORD EMBEDDINGS EXPLOITING MORPHOLOGY}
% \newcommand{\TEZBASLIK}{SEMANTIC RELATION EXTRACTION BY ENRICHING MORPHOLOGICAL WORD EMBEDDINGS FOR TURKISH}

\newcommand{\TEZBASLIKTR}{THESIS BAŞLIK}

\newcommand{\ME}{GÖKHAN ERCAN}

%----------- TODO RESOURCES----------------------
%http://ftp.ntua.gr/mirror/ctan/macros/latex/contrib/todonotes/todonotes.pdf
%Commands: unsure,change,info,improve,askadv
\usepackage[pdftex,dvipsnames]{xcolor}  %Coloured text etc.

% cancel için - üzerini çizmek (strikethrough)
\usepackage{calc}
\newsavebox\CBox
\newcommand\hcancel[2][0.5pt]{%
  \ifmmode\sbox\CBox{$#2$}\else\sbox\CBox{#2}\fi%
  \makebox[0pt][l]{\usebox\CBox}%  
  \rule[0.5\ht\CBox-#1/2]{\wd\CBox}{#1}}
  
%citet COLI'den geliyormuş.
%\renewcommand{\citet}[1]{%
%	\cite{#1}%
%}
%\renewcommand{\citep}[1]{%
%	\cite{#1}%
%}
\renewcommand{\cite}{\citep}
\usepackage[utf8]{inputenc}

%mailto için
\usepackage{hyperref} 

%todonotes performans sebebiyle remove edildi. 6'dan 3sn'ye düştü. 
%https://tex.stackexchange.com/questions/8791/speeding-up-latex-compilation

%For Matrix Images (imported from https://www.overleaf.com/project/63d5677f38388e1fb22f2853)
\usepackage{graphicx}
%\usepackage[export]{adjustbox}
\PassOptionsToPackage{export}{adjustbox}  %this is the fix.


%TIKS
\usepackage{tikz}                    % in preamble. todolist lib yoksa bu gerekli. 
%\usepgflibrary{patterns} % LATEX and plain TEX and pure pgf
%\usepgflibrary[patterns] % ConTEXt and pure pgf
\usetikzlibrary{patterns} % LATEX and plain TEX when using Tik Z
\usetikzlibrary[patterns] % ConTEXt when using Tik Z

\usepackage{amsmath} %Sim-Rel space denklemi için.
\usepackage{lscape}   %landspace page yapabilmek için.
\usepackage{adjustbox}  %tablo oto genişlesin ???

\newcommand{\cout}[1]{\ignorespaces} %cout: commenting out a word ref: https://tex.stackexchange.com/questions/276697/commenting-out-a-few-words-within-a-paragraph

%FOOTNOTE
%Bu hata verdi COLI'de!
%\usepackage{scrextend}  
%aynı footnote'a birden fazla ref verebilmek için. https://tex.stackexchange.com/questions/35043/reference-different-places-to-the-same-footnote

%-----MULTILINE TABLE-------
\usepackage{pbox}
% Tablo hücresi içinde alt satır için: https://tex.stackexchange.com/questions/2441/how-to-add-a-forced-line-break-inside-a-table-cell

\usepackage{booktabs, multirow} % for borders and merged ranges

%HIDING COLUMNS: https://texblog.org/2014/10/24/removinghiding-a-column-in-a-latex-table/
\usepackage{array}
\newcolumntype{H}{>{\setbox0=\hbox\bgroup}c<{\egroup}@{}}


%Cancel-out / Strikethrough için https://tex.stackexchange.com/questions/23711/strikethrough-text
\usepackage{cancel}

\usepackage{xfp}

%Birden fazla equation verebilmek için
%ref:https://ftp.cc.uoc.gr/mirrors/CTAN/macros/latex/contrib/cleveref/cleveref.pdf
%\usepackage{cleveref}	%Kullanmıyorum

%\usepackage{graphicx}
%\usepackage[export]{adjustbox}
%\PassOptionsToPackage{export}{adjustbox}

\definecolor{myblue}{rgb}{0.26,0.52,0.96}
\definecolor{myred}{rgb}{1,0,0}
\definecolor{myyel}{rgb}{0.96,0.7,0}
\definecolor{mygreen}{rgb}{0.458,0.768,0.49}

%for numberless footnote
\newcommand\blfootnote[1]{%
	\begingroup
	\renewcommand\thefootnote{}\footnote{#1}%
	\addtocounter{footnote}{-1}%
	\endgroup
}

%Verbatim block'ları resize edebilmek formatlayabilmek için
\usepackage{adjustbox}
\usepackage{varwidth}

\usepackage[none]{hyphenat}

\usepackage{chngcntr}

\usepackage{pdflscape}	%yatay sayfa için

\newcommand{\lt}{\textless}		%sn template'inde < > bunlar çalışmadı.
\newcommand{\gt}{\textgreater}

%\usepackage{natbib}




\hypersetup{urlcolor=blue, colorlinks=true}  

\usepackage{multirow}
\usepackage{sectsty}
\usepackage{multirow}
\usepackage{marginnote}
\usepackage{framed}

%% ----------------------------------------------------------------
\renewcommand\contentsname{\large{TABLE OF CONTENTS}}
\renewcommand\listfigurename{\large{LIST OF FIGURES}}
\renewcommand\listtablename{\large {LIST OF TABLES}}
\renewcommand\bibname{REFERENCES}

\chapterfont{\large\centering\bfseries}
\chaptertitlefont{\large\centering\bfseries}
\sectionfont{\normalsize\bfseries}
\subsectionfont{\normalsize\bfseries}
\subsubsectionfont{\normalsize\bfseries}

\hypersetup{pdfborder={0 0 0}, colorlinks=false}  % Black hyperlinks.

\renewcommand{\figurename}{\normalsize Figure}
% Different font in captions
\newcommand{\captionfonts}{\normalsize}
\makeatletter  % Allow the use of @ in command names
\long\def\@makecaption#1#2{%
  \vskip\abovecaptionskip
  \sbox\@tempboxa{{\captionfonts #1: #2}}%
  \ifdim \wd\@tempboxa >\hsize
    {\captionfonts #1: #2\par}
  \else
    \hbox to\hsize{\hfil\box\@tempboxa\hfil}%
  \fi
  \vskip\belowcaptionskip}
\makeatother   % Cancel the effect of \makeatletter




%BU HACK'İ YENİ TEZ FORMATI İÇİN YAPIYORUZ
\usepackage{titlesec} % Load titlesec to help format titles
\renewcommand{\chaptername}{Chapter} % Define the chapter prefix
\usepackage{titlesec}
\titleformat{\chapter}[display]
{\normalfont\large\bfseries\centering} % Formatting the chapter, including centering
{CHAPTER \thechapter} % Adding "Chapter" followed by the number
{20pt} % Space between number and title
{\large} % Style for the chapter title (this will also be centered)

\usepackage{booktabs}	%Custom toprule formatı verilebiliyor. 
\usepackage{caption}

\sloppy                              % Relaxes line breaking rules
%\setlength{emergencystretch}{1em}    % Adds extra flexibility to line spacing  olmuyor. 


\begin{document}

\captionsetup{skip=10pt,labelfont=bf,justification=raggedright,singlelinecheck=false} % Adjust the space as needed

\thispagestyle{empty}
\vspace*{-4cm}

\reversemarginpar
\marginnote{
%\begin{framed}
%\vspace*{2cm}
%\rotatebox{270}{\ME}
%\\[7cm] 
% \rotatebox{270}{PhD Thesis}
%\\[9cm] 
% \rotatebox{270}{2024}
%\\[2cm] %
%\end{framed}
 }

%{\bfseries
 
\begin{center}	
\vspace*{3.5cm}

\begin{center}
\large{T.C.} \\
\large{IŞIK UNIVERSTIY} \\
\large{SCHOOL OF GRADUATE STUDIES}
\end{center}

\vspace*{2cm}

\large{DOCTORAL THESIS} \\
\large{DEPARTMENT OF COMPUTER ENGINEERING} \\
%\large{DOCTOR OF…………………….. PROGRAM} \\

\vspace*{2cm}

\large{Gökhan ERCAN}
\vspace{1\baselineskip}		%Ogrenci No placeholder.

\vspace*{2cm}

\large{
	THESIS TITLE
}

\vspace*{2cm}

\large{
	SUPERVISOR
}\\
\large{Prof. Dr. Olcay Taner YILDIZ}
%\\[8cm] %
%\large{\ME} 
%\\[10cm] %
%I\c{S}IK UNIVERSITY
%\\

%\\[8cm] %

\vspace*{4cm}
İSTANBUL, September 2024
%\\[2cm] %

\end{center}

%} % End the bold section
\clearpage
\input{titlePage2}
\frontmatter     
\thispagestyle{empty}
%{\bfseries
\begin{center}
	\vspace*{-0.9cm}
%	I\c{S}IK UNIVERSITY \\ 
%	GRADUATE SCHOOL OF SCIENCE AND ENGINEERING \\[4cm]

\begin{center}
	\large{T.C.} \\
	\large{IŞIK UNIVERSTIY} \\
	\large{SCHOOL OF GRADUATE STUDIES}
\end{center}

\vspace*{0.5cm}

\large{DOCTORAL THESIS} \\
\large{DEPARTMENT OF COMPUTER ENGINEERING} \\
%\large{DOCTOR OF…………………….. PROGRAM} \\

\vspace*{1.5cm}

\large{Gökhan ERCAN}	\\
\large{(214DCS8012)}

\vspace*{1.5cm}

\large{
	THESIS TITLE
}
%	\TEZBASLIK\\[3cm]
%	\ME\\[3cm]
\end{center} 

\begin{flushleft}
%	APPROVED BY:\\[1.0cm] %1.5
	Date:
	\makebox[13.2cm][r]{Signature}
	\newline
	
	Thesis Supervisor:\\ 
	\makebox[6.5cm][l]{Prof. Dr. Olcay Taner YILDIZ}
	\makebox[4cm][l]{Özyeğin University}
	
%	\hspace{1.5cm}\rule[0em]{8em}{0.5pt}
%	\\[0.2cm](Thesis Supervisor)\\[1cm]
	
	\vspace*{0.2cm}
	Jury Members: \\
	\makebox[6.5cm][l]{Member 1}
	\makebox[4cm][l]{University}
	\vspace*{0.2cm}
	\\
%	\makebox[6.5cm][l]{Dr. Faik Boray TEK}
%	\makebox[4cm][l]{İstanbul Technical University}
%	\vspace*{0.2cm}
%	\\
	\makebox[6.5cm][l]{Member 2}
	\makebox[4cm][l]{University}
	\vspace*{0.2cm}
	\\
	\makebox[6.5cm][l]{Member 3}
	\makebox[4cm][l]{University}
	\vspace*{0.2cm}
	\\
	\makebox[6.5cm][l]{Member 4}
	\makebox[4cm][l]{University}
	\vspace*{0.2cm}

	
	\vspace*{2cm}
	
	\begin{center}	
	İSTANBUL, September 2024
	\end{center}
%	APPROVAL DATE:
%	\hspace{2.2cm} ..../..../....
\end{flushleft}

%} % End the bold section

\setmarginsrb           {4cm}  % left margin
{2.5cm}  % top margin
{2.5cm}  % right margin
{2.5cm}  % bottom margin
{0cm}  % head height
{0cm}  % head sep
{0cm}  % foot height
{1.3cm}  % foot sep
\clearpage

\addtotoc{ABSTRACT}
\abstract{ 

% Abstract-P1
\noindent abstract1
% Abstract-P2
\noindent abstract2

% Abstract-P3
abstract3

\noindent \textbf {\small{Keywords: }}
%% ---TÜRKÇE ÖZET-------------------------------------------------------------

\clearpage 
\addtotoc{ÖZET}
\ozet { 
\noindent tr özet
\noindent \textbf {\small{Anahtar kelimeler: }}



%% ----------------------------------------------------------------
% Acknowledgements page
%\acknowledgements{
% \addtocontents{toc}{\vspace{1em}}  
% \noindent This study was supported by The Scientific and
% Technological Research Council of Turkey (T\"{U}B\.{I}TAK) Grant No: 
% }
\clearpage  

%% ----------------------------------------------------------------

% Dedicated to
% \pagestyle{empty}  
% \dedicatory{To\ldots}

%% ----------------------------------------------------------------

\renewcommand{\contentsname}{TABLE OF CONTENTS}
\addtocontents{toc}{\protect\thispagestyle{empty}}
\tableofcontents  % Write out the Table of Contents


%% ----------------------------------------------------------------

\listoftables  % Write out the List of Tables

%% ----------------------------------------------------------------

\listoffigures  % Write out the List of Figures

%% ----------------------------------------------------------------
\setstretch{1.5}  
\clearpage  % Start a new page
\listofsymbols{ll}  % Include a list of Abbreviations (a table of two columns)
{
	\textbf{SYM} & symbol1 \\
}



%% ---------------CHAPTER 1-------------------------------------------------
\mainmatter   
\pagestyle{plain}
\chapter{INTRODUCTION}
\label{ch:introduction}
intro text

\section{SECTION}
section text

Sample cite:   \cite{mikolov2013distributed}.
Sample cite p: \citep{mikolov2013efficient}.
Sample cite  t:   \citet{mikolov2013efficient}

\subsection{Subsection}
Subsection text

\subsubsection{SubSubsection}
Subsubsection text

\paragraph{Paragraph1}
p text
\paragraph{Paragraph2}
p text

















%--------------------ENDINGS PARTS-------------------------------------------
\chapter*{APPENDICES}
\addtotoc{APPENDICES}
%\appendix				%MIT style
%\newpage
%\begin{appendices}		%ELRA style
\label{appendix}
	
%SECTION1-----------------------------------------------------
\section*{APPENDIX A}
\label{secA1}

\newpage
\section*{APPENDIX B}
\label{secB1}

\clearpage 
\renewcommand\bibname{REFERENCES}
\addtotoc{REFERENCES}
\bibliographystyle{abbrvnat_custom}		%Thesis.(aux,blg,bbl) dosyalarını silmek gerekiyor önce. Taktı mı derlemiyor bi türlü. cache'leri silip pdflatex ve bibtex run et.
%\bibliographystyle{IEEE}
\bibliography{myBibliography.bib}

\end{document}